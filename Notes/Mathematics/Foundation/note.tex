% Created 2019-05-30 Thu 17:16
% Intended LaTeX compiler: xelatex
\documentclass[11pt, a4paper]{article}
\usepackage{graphicx}
\usepackage{grffile}
\usepackage{longtable}
\usepackage{wrapfig}
\usepackage{rotating}
\usepackage[normalem]{ulem}
\usepackage{amsmath}
\usepackage{textcomp}
\usepackage{amssymb}
\usepackage{capt-of}
\usepackage{hyperref}
\usepackage{fontspec}
\setmainfont{EB Garamond}
\usepackage[margin=20mm]{geometry}
\usepackage{amsthm}
\renewcommand\qedsymbol{$\blacksquare$}
\newtheorem{theorem}{Theorem}
\newtheorem{lemma}[theorem]{Lemma}
\newtheorem{corollary}[theorem]{Corollary}
\author{Ahmed Khaled}
\date{\today}
\title{Mathematical Foundation}
\hypersetup{
 pdfauthor={Ahmed Khaled},
 pdftitle={Mathematical Foundation},
 pdfkeywords={},
 pdfsubject={},
 pdfcreator={Emacs 27.0.50 (Org mode 9.2.3)}, 
 pdflang={English}}
\begin{document}

\maketitle
\begin{abstract}
This Document contain my notes about Axioms, Definitions and basic theories.
\end{abstract}

\section{Real Numbers}
\label{sec:org2f5e07d}
In rigorous mathematics real number is a set of numbers defined as a complete, ordered field

\subsection{Fields}
\label{sec:org061704e}

\begin{itemize}
\item \textsc{Def}. \emph{Field} is a non-empty set on which two \(\overline{\mbox{binary operation}}\) are
defined \marginpar{refer to group theory and set theory}

\item \textsc{Def}. \emph{Binary Operation} in field \(\mathbb{F}\) is a function that "take"
an ordered pair of element and "return" an element in \(\mathbb{F}\), and it said to be
the operation on the set whose both domain and co-domain in the same set.
\end{itemize}
\[ \forall a,b \in \mathbb{F} (\exists c \in \mathbb{F}) : (c = a \circ b) \]


\begin{itemize}
\item the 9 golden basic most primitive axioms:
\begin{enumerate}
\item \textsc{Axi}. \emph{Associative law for addition} \(( a + b) + c = a + ( a + c )\)
\item \textsc{Axi}. \emph{Existence of additive identity} \(\exists 0:  a + 0 = 0 + a = a\)
\item \textsc{Axi}. \emph{Existence of additive inverse} \(\forall a \in \mathbb{R} \exists (-a) : a + (-a) = (-a) + a = 0\)
\item \textsc{Axi}. \emph{Commutative law of addition} \(a + b = b + a\)
\item \textsc{Axi}. \emph{Associative law for multiplication} \(( a \cdot b) \cdot c = a \cdot ( a \cdot c )\)
\item \textsc{Axi}. \emph{Existence of multiplicative identity} \(\exists 1 \neq 0:  a \cdot 1 = 1 \cdot a = a\)
\item \textsc{Axi}. \emph{Existence of multiplicative inverse} \(\forall a \neq 0 \in \mathbb{R} \exists (a^{-1}) : a + (a^{-1}) = (a^{-1}) + a = 0\)
\item \textsc{Axi}. \emph{Commutative law of multiplication} \(a \cdot b = b \cdot a\)
\item \textsc{Axi}. \emph{Distributive law} \(a \cdot ( b + c ) = a \cdot b + a \cdot c\)
\end{enumerate}

\item Theorem
\end{itemize}

\begin{theorem}
  $ \forall a \in \mathbb{F}: a \cdot 0 = 0 $
\end{theorem}

\begin{proof}
  using axiom Num.9
  \begin{align*}
    a \cdot 0 &= a \cdot (0 + 0) \\
          &= a \cdot 0 + a \cdot 0 \\
  \end{align*}
by adding $-(a \cdot 0)$ to both side
\[ a \cdot 0 = 0 \]
\end{proof}
\subsection{Order}
\label{sec:orgefcb8ad}

\begin{itemize}
\item \textsc{Def}. \emph{Ordered field} \(\mathbb{F}\). A field is said to be ordered if it has a distinguished subset
\(\overline{P \subset \mathbb{F}}\) \marginpar{Positive Numbers}, that have the follow properties:
\begin{enumerate}
\item \emph{Trichotomy}: which mean every element \(a \in \mathbb{F}\) satisfied one and only one of the follow
\begin{enumerate}
\item \(a \in P\)
\item \(-a \in P\)
\item \(a = 0\)
\end{enumerate}
\item \emph{Closure under addition} \(\forall a,b ( a,b \in P \implies a + b \in P)\)
\item \emph{Closure under multiplication} \(\forall a, b ( a, b \in P \implies a \cdot b \in P)\)
\end{enumerate}
and we said that \(a < b\) means \(b - a \in P\). which is clear if \(b\) bigger than \(a\) then the difference
between them is positive number.

\item Theorem

\begin{theorem}
  $ \forall a, b \in \mathbb{F}$ one fo the following hold \\
1. $a<b$
2. $a>b$
3. $a=b$
\end{theorem}
\end{itemize}

\begin{proof}
    using Trichotomy one of these hold
  1. $a, b \in P$ then it either
     1. $a - b \in P$ then we say $a < b$
     2. $b - a \in P$ then we say $a > b$
  2. $a, -b \in p$ then by using Closure under addition $a + (-b) \in P$ then we say $a > b$
  3. the opposite of Num.2
\end{proof}

\begin{theorem}
  $a < b \implies a + c < b + c$
\end{theorem}

\begin{proof}
  Suppose $a + c < b + c$, then it means $a + c -( b + c) \in P$ which deduce to $a < b$
\end{proof}

\begin{theorem}
  \textit{Transitivity}. $a < b \land b < c \implies a < c$
\end{theorem}

\begin{proof}
  $a < b$ means $b - a \in P$ and $b < c$ means $c - b \in P$. Thuh, $c - b - a + b \in P$ which means
$a < c$
\end{proof}

\begin{theorem}
  $a, b < 0 \implies a \cdot b > 0$
\end{theorem}

\begin{proof}
  Suppose $a, b < 0$ then \( -a \cdot -b \in P \), thuh $a \cdot b > 0$
\end{proof}

\begin{corollary}
  $\forall a \neq 0 : a \cdot a \equiv a^{2} > 0$
\end{corollary}

\begin{proof}
  There is two cases
 1. $a > 0$ in this case $a^{2} > 0$ by closure under multiplication
 2. $a < 0$ is a spicial case from previous theorem when $a = b$
\end{proof}
\end{document}